\documentclass[12pt,a4paper,openright,twoside]{book}
\usepackage[utf8]{inputenc}
\usepackage{disi-thesis}
\usepackage{code-lstlistings}
\usepackage{notes}
\usepackage{shortcuts}
\usepackage{acronym}
\usepackage[acronym]{glossaries}

\school{\unibo}
\programme{Corso di Laurea Triennale in Ingegneria e Scienze Informatiche}
\title{DGA Domain Generation Algorithm}
\author{Simone Collorà}
\date{\today}
\subject{Programmazione a Oggetti}
\supervisor{Prof. Mirko Viroli}
\cosupervisor{Dott. CoSupervisor 1}
\morecosupervisor{Dott. CoSupervisor 2}
\session{I}
\academicyear{2024-2025}

% Definition of acronyms
\acrodef{IoT}{Internet of Thing}
\acrodef{vm}[VM]{Virtual Machine}
\newacronym{DGA}{DGA}{Domain Generation Algorithm}
\newacronym{C&C}{C\&C}{Command and Control}


\mainlinespacing{1.241} % line spacing in mainmatter, comment to default (1)

\begin{document}

\frontmatter\frontispiece

\begin{abstract}	
Max 2000 characters, strict.
\end{abstract}

\begin{dedication} % this is optional
Optional. Max a few lines.
\end{dedication}

%----------------------------------------------------------------------------------------
\tableofcontents   
\listoffigures     % (optional) comment if empty
\lstlistoflistings % (optional) comment if empty
%----------------------------------------------------------------------------------------

\mainmatter

%----------------------------------------------------------------------------------------
\chapter{Introduction}
\label{chap:introduction}
%----------------------------------------------------------------------------------------

Write your intro here.
\sidenote{Add sidenotes in this way. They are named after the author of the thesis}

La sicurezza informatica in particolar modo via internet è un argomento di crescente importanza
nel mondo moderno. Con il passare del tempo,
i sistemi di protezione sono diventati sempre più sofisticati
e potenti ma allo stesso tempo anche gli hackers 
hanno sviluppato tecniche sempre più avanzate per eludere i sistemi di protezione.
Tra queste vi è sicuramente l'uso 
dei \acrfull{C&C} servers. I \acrshort{C&C} sono sistemi che permettono di controllare
i computer infetti da un virus o un malware. Per nascondere questi attacchi
vengono spesso usati i \acrfull{DGA}, algoritmi in grado di generare
nomi di dominio pseudo casuali.



\paragraph{Structure of the Thesis}

\note{At the end, describe the structure of the paper}

\chapter{State of the art}

I suggest referencing stuff as follows: \cref{fig:random-image} or \Cref{fig:random-image}

\begin{figure}
    \centering
    \includegraphics[width=.8\linewidth]{figures/random-image.pdf}
    \caption{Some random image}
    \label{fig:random-image}
\end{figure}

\section{Some cool topic}

\chapter{Contribution}

You may also put some code snippet (which is NOT float by default), eg: \cref{lst:random-code}.

\lstinputlisting[float,language=Java,label={lst:random-code}]{listings/HelloWorld.java}

\section{Fancy formulas here}

%----------------------------------------------------------------------------------------
% BIBLIOGRAPHY
%----------------------------------------------------------------------------------------

\backmatter

\nocite{*} % Remove this as soon as you have the first citation

\bibliographystyle{alpha}
\bibliography{bibliography}

\begin{acknowledgements} % this is optional
Optional. Max 1 page.
\end{acknowledgements}

\end{document}
